\documentclass[12pt]{article}
\author{Alex Ho}
\title{FYS4150 - Computational Physics \\ Project 1}
\usepackage{listings}
\usepackage{graphicx}
\usepackage{verbatim}
\usepackage{amsmath}
\usepackage[utf8]{inputenc}
\usepackage[english, norsk]{babel}
\usepackage{xcolor}
\usepackage{hyperref}

\lstset{
language=Python,
basicstyle=\ttfamily,
otherkeywords={self},             
keywordstyle=\ttfamily\color{red!90!black},
keywords=[2]{True,False},
%keywords=[3]{ttk},
keywordstyle={[2]\ttfamily\color{red!80!orange}},
emph={MyClass,__init__},          
emphstyle=\ttfamily\color{red!80!black},    
stringstyle=\color{red!70!yellow},
showstringspaces=false,
commentstyle=\color{blue!90!black},
breaklines=true,
tabsize=3,
moredelim=**[is][\color{blue}]{@}{@}
}

\begin{document}
\maketitle
\section{Introduction}
We will in this project solve the one-dimensional Poisson equation, given as
\begin{align}
-u''(x) = f(x)
\end{align}
numerically, with Dirichlet boundary conditions by rewriting the Poisson equation to a set of linear equations. In this project, we will also focus on memory allocation and floating point operations in our numerical algorithm.
\section{Method}
\subsection*{a)}
We can discretize the Poisson equation 
\section{Implementation}
\section{Results}
\section{Conclusion}
\end{document}